\documentclass[a4paper]{article}

\usepackage{listings}
\usepackage{amsthm}
\usepackage{mathtools}

\DeclarePairedDelimiter\floor{\lfloor}{\rfloor}
\DeclarePairedDelimiter\ceil{\lceil}{\rceil}
\DeclarePairedDelimiter\abs{\lvert}{\rvert}
\DeclareMathOperator{\length}{length}
\DeclareMathOperator{\rnd}{rnd}
\DeclareMathOperator{\fsqrt}{fsqrt}
\DeclareMathOperator{\fsqr}{fsqr}
\DeclareMathOperator{\float}{float}
\DeclareMathOperator{\rint}{rint}

\theoremstyle{plain}
\newtheorem{theorem}{Theorem}
\newtheorem{lemma}[theorem]{Lemma}
\newtheorem{corollary}[theorem]{Corollary}

\theoremstyle{definition}
\newtheorem{definition}[theorem]{Definition}
\newtheorem{example}[theorem]{Example}

\title{Python's integer square root algorithm}
\author{Mark Dickinson}

\begin{document}
\lstset{language=Python}
\maketitle
\begin{abstract}
We present an adaptive-precision variant of Heron's method for computing
integer square roots of arbitrary-precision integers. The method is efficient
both at small scales and asymptotically, and represents an attractive
compromise between speed and simplicity. Since Python 3.8, the algorithm is
used by the CPython implementation of the Python programming language for its
standard library integer square root function.
\end{abstract}
\section{Introduction}

We start with a simple definition.

\begin{definition}
  For a nonnegative integer $n$, the \emph{integer square root} of $n$ is
  the unique nonnegative integer $a$ satifying $a^2 \le n < (a + 1)^2$.
\end{definition}

Equivalently, the integer square root of~$n$ is the integer part of the exact
square root of~$n$, or $\floor{\sqrt n}$.

The integer square root is a basic building block of any arbitrary-precision
arithmetic toolkit. Many number-theoretic and cryptographic algorithms require
the ability to detect whether a given integer is a square, and if so, to
extract its root.

For small $n$, it's feasible to use floating-point arithmetic to
compute integer square roots. For example, assuming IEEE 754 binary64 format
floating-point and a correctly-rounded square root operation, one can show that
computing $\floor{\sqrt n}$ directly gives the integer square root of $n$,
provided that $n < 2^{52} + 2^{27}$. However, neither the Python language nor
the CPython reference implementation of that language provides a guarantee of
either IEEE 754 format floating-point or correct rounding of floating-point
arithmetic operations. As such, any floating-point-based method for computing
an integer square root would need a correctness check along with a pure-integer
fallback method for the case where the floating-point square root operation
fails to provide sufficient accuracy. To avoid this complication, and to allow
integer square roots to be computed for arbitrarily large integers, it's
desirable to have an integer square root algorithm that works entirely with
integer arithmetic.

In this article we present a simple and fast pure-integer algorithm to compute
the integer square root of an arbitrary positive integer. This algorithm has
been implemented for CPython's \lstinline{math} module and is available from
Python 3.8 onwards as \lstinline{math.isqrt}.

Section~\ref{old_method} of this article reviews a well-known method for
computing integer square roots based on Heron's method. This provides much of
the background we need to introduce our algorithm, which is presented in
section~\ref{new_method}.

\section{Heron's method}
\label{old_method}

In this section we describe a well-known approach to computing integer square
roots, based on an adaptation of Heron's method to the domain of positive
integers.

Heron's method, also known as the Babylonian method, is a procedure for
approximating the square root of a given positive real number~$t$. It's based
on the idea that if $x$ is a positive real approximation to~$\sqrt t$, then
$$h(x) = \frac{x + t/x}2$$ is an improved approximation. Applying~$h$ to that
improved approximation yields a still better approximation~$h(h(x))$, and so
on. It can be shown that the sequence of iterates of~$h$ on~$x$ converges
towards~$\sqrt t$ from any positive initial value~$x$, and that once the
sequence gets sufficiently close to~$\sqrt t$, convergence is
quadratic, so that the number of correct decimal places roughly doubles with
each iteration.

\begin{example}
  Suppose $t=2$. Given an approximation $7/5 = 1.4$ to the square root
  $1.414213562\dots$ of $t$, a single iteration of Heron's method produces a
  new approximation, accurate to $4$ decimal places after the point:
  $$h(7/5) = \frac{7/5 + 2/(7/5)}2 = 99/70 = 1.414285714\dots.$$ Applying a
  second iteration with $99/70$ as input gives
  $$h(h(7/5)) = h(99/70) = 19601/13860 =
  1.414213564\dots,$$ which is accurate to $8$ decimal places. A third
  iteration gives a value accurate to $17$ decimal places.
\end{example}

Heron's method can be adapted to the domain of positive integers. For the
remainder of this section we assume that $n$ is a fixed positive integer, and
we define a function $g$ on the positive integers by
$$g(a) = \floor*{\frac{a + \floor{n/a}}2}.$$

The idea is that, just like its real counterpart~$h$, the function $g$ should
transform a poor approximation to the integer square root of $n$ into an
improved approximation, and so by applying $g$ repeatedly we hope to eventually
reach the integer square root of~$n$. This works, but we have to be a little
careful with the details, and in particular the termination condition. The
lemmas below make this precise.

\begin{lemma}\label{heron_high}
  For any positive integer $a$, $\floor{\sqrt n} \le g(a)$.
\end{lemma}

\begin{proof}
  Expanding and rearranging the inequality $0 \le (a - \sqrt n)^2$ gives
  $$2a\sqrt n \le a^2 + n.$$
  Dividing through by $2a$ gives
  $$\sqrt n \le \frac{a + n/a}2,$$
  which can also be seen as the AM-GM inequality applied to~$a$ and~$n/a$.
  Hence
  $$\floor{\sqrt n} \le \floor*{\frac{a + n/a}2}
  = \floor*{\frac{\floor{a + n/a}}2}
  = \floor*{\frac{a + \floor{n/a}}2} = g(a).$$
\end{proof}

\begin{lemma}\label{heron_decreases}
  Suppose~$a$ is a positive integer satisfying~$\floor{\sqrt n} < a$. Then
  $g(a) < a$.
\end{lemma}

\begin{proof}
  We have the following chain of equivalences:
  \begin{align*}
    \floor{\sqrt n} < a &\iff \sqrt n < a \\
                        &\iff n < a^2 \\
                        &\iff n/a < a \\
                        &\iff \floor{n/a} < a \\
                        &\iff a + \floor{n/a} < 2a \\
                        &\iff \frac{a + \floor{n/a}}2 < a \\
                        &\iff \floor*{\frac{a + \floor{n/a}}2} < a \\
                        &\iff g(a) < a.
  \end{align*}
  This completes the proof.
\end{proof}

Now let~$a$ be any integer satisfying $\floor{\sqrt n} \le a$, and let
$$a_0, a_1, a_2, \dots$$ be the sequence of iterates of $g$ applied to $a$;
that is, $a_0 = a$ and for all~$i \ge 0$, $a_{i+1} = g(a_i)$. We'll show that
this sequence must eventially reach~$\floor{\sqrt n}$.

\begin{lemma}\label{sequence_high}
  For all $i\ge 0$, $\floor{\sqrt n} \le a_i$.
\end{lemma}
\begin{proof}
  This follows directly from Lemma~\ref{heron_high} for $i\ge 1$, and
  from our assumption on~$a$ for $i=0$.
\end{proof}

\begin{lemma}\label{isqrt_condition}
  For all $i\ge 0$, $\floor{\sqrt n} = a_i$ if and only if $a_i\le a_{i+1}$.
\end{lemma}

\begin{proof}
  By Lemma~\ref{sequence_high}, either $\floor{\sqrt n} < a_i$ or $\floor{\sqrt
  n} = a_i$. If $\floor{\sqrt n} = a_i$ then by Lemma~\ref{sequence_high}
  again, $a_i \le a_{i+1}$. If $\floor{\sqrt n} < a_i$ then $a_{i+1} < a_i$ by
  Lemma~\ref{heron_decreases}.
\end{proof}

\begin{theorem}
  There exists an~$i\ge 0$ such that $a_i$ is the integer square root of~$n$.
\end{theorem}

\begin{proof}
  By the well-ordering principle, there's an~$i$ for which $a_i$ is minimal
  amongst all elements of the sequence. That minimality implies $a_i \le
  a_{i+1}$, and so by Lemma~\ref{isqrt_condition}, $a_i$ is the integer square
  root of~$n$.
\end{proof}

Note that while the sequence is guaranteed to encounter $\floor{\sqrt n}$
eventually, it's \emph{not} necessarily true that the sequence \emph{converges}
to~$\floor{\sqrt n}$, in the sense that it's eventually constant. For
example, if $n=15$, the sequence of iterates of $g$ eventually alternates
between $3$ and $4$, and more generally a similar alternation will occur for
any $n$ of the form $a^2 - 1$ for some $a \ge 2$.

By the results above, we can find $\floor{\sqrt n}$ by taking any starting
point $a \ge \floor{\sqrt n}$, and replacing $a$ with $g(a)$ until $a\le g(a)$.
Listing~\ref{isqrt_heron_direct} expresses this algorithm in Python, using
$a=n$ as the starting point for the iteration. Note that the function
\lstinline$isqrt$ assumes $n > 0$. Extending the code to return~$0$ for $n=0$
and to raise an exception for negative~$n$ is left as an easy exercise for the
reader.

A linguistic note for readers unfamiliar with Python: the \lstinline$//$
operator performs ``floor division'': \lstinline$n//a$ represents $\floor{n /
a}$. Indeed, from a computational perspective, an expression like $\floor{n /
a}$ is misleading: it resembles a composition of two operations, while most
programming languages or arbitrary-precision integer-arithmetic packages will
provide some form of integer division as an integer-to-integer primitive.

\lstinputlisting[
  float,
  label=isqrt_heron_direct,
  frame=single,
  caption={Integer square root via Heron's method, version 1},
  lastline=12]{isqrt_heron_direct.py}

The use of~$n$ as an initial estimate is inefficient for large~$n$: the initial
iterations will roughly halve the estimate each time, and many iterations will
be needed to get into the general neighborhood of the square root. It would be
better to choose a starting value~$a$ that's closer to $\sqrt n$. If we can't
guarantee that $a\ge \floor{\sqrt n}$, we can replace $a$ with $g(a)$ before
running the main algorithm: by Lemma~\ref{heron_high}, $g(a)$ is guaranteed to
satisfy $g(a) \ge \floor{\sqrt n}$.

One simple and reasonably efficient possibility for the starting value is to
use the smallest power of two exceeding $\floor{\sqrt n}$. Before giving the
code, we make a definition.

\begin{definition}
  For a nonnegative integer $n$, the \emph{bit length} of $n$, which we'll
  write simply as $\length(n)$, is the least nonnegative integer $e$ for which
  $n < 2^e$.
\end{definition}

For positive $n$, the bit length of $n$ is $1 + \floor{\log_2(n)}$, while the
bit length of zero is zero. In Python, the bit length of a nonnegative integer
\lstinline$n$ can be computed as \lstinline$n.bit_length()$.

Given nonnegative integers $n$ and $e$, we have the following chain
of equivalences:
\begin{align*}
  \floor{\sqrt n} < 2^e
    &\iff \sqrt n < 2^e \\
    &\iff n < 2^{2e} \\
    &\iff \length(n) \le 2e \\
    &\iff \length(n) < 2e + 1 \\
    &\iff \floor{(\length(n) - 1) / 2} < e \\
    &\iff \floor{(\length(n) + 1) / 2} \le e
\end{align*}
So taking $e = \floor{(\length(n) + 1) / 2}$, $2^e$ is the smallest power of
two that strictly exceeds the square root of~$n$. Using that as our starting
guess, and taking the opportunity to streamline the previous code a little at
the same time, we get the algorithm in Listing~\ref{isqrt_heron_improved}, in
which the termination condition $a \le g(a)$ has been replaced with the
equivalent condition $a \le \floor{n/a}$.

\lstinputlisting[
  float,
  label=isqrt_heron_improved,
  frame=single,
  caption={Integer square root via Heron's method, streamlined},
  lastline=7]{isqrt_heron_improved.py}

We chose $e$ to satisfy $\floor{\sqrt n} < 2^e$. For correctness, we only need
to satisfy the weaker condition $\floor{\sqrt n} \le 2^e$. Given that, we could
tighten our initial bound slightly by using $\length(n-1)$ in place of
$\length(n)$. But that costs an extra operation for all inputs $n$, for a
benefit only in the rare case that $n$ is an exact power of four. As such, the
change is probably not valuable.

The algorithm shown in Listing~\ref{isqrt_heron_improved} is well-known and
well-used. It has the virtues of simplicity and easy-to-establish correctness,
and is reasonably efficient for inputs that aren't too large. Nevertheless,
there are at least three areas in which there's room for improvement. First,
for large inputs (say a few thousand bits or more), the first few iterations of
the algorithm are performing expensive full-precision divisions to obtain only
a handful of new correct bits at each iteration. Second, our particular choice
of initial guess is somewhat ad hoc, and could probably be improved, possibly
at the expense of some additional complexity: in that sense, the algorithm
feels incomplete, or at least unsatisfyingly inelegant---it's really the
combination of two separate algorithms: one to find an initial guess, and a
second to improve that guess until we hit the integer square root. Finally,
there's potential inefficiency towards the end of the algorithm, and in
particular the fact that the termination condition requires that we perform an
extra division \emph{after} we've reached the desired result seems like an
unnecessary inefficiency. To illustrate the last point, consider the following
example.

\begin{example}
  Take $n = 16785408$, which is just a little larger than $2^{24} = 16777216$.
  Our starting guess for the square root is $2^{13} = 8192$. Successive
  iterations give $5120$, $4199$, $4098$, $4097$, and finally $4096$, which
  is the integer square root. Each iteration requires one division, and
  a final division is needed to establish the termination condition, for
  a total of six divisions.
\end{example}

So even starting from $4098$, just a distance of two away from the true integer
square root, three more divisions are required before the correct integer
square root can be identified and returned. Note that this example is not
typical: it was deliberately chosen to show worst-case behaviour.

The algorithm introduced in the following section addresses all three of these
deficiencies, at the expense of only a little extra complexity.

\section{Variable-precision Heron's method}
\label{new_method}

In this section we introduce a simple variant of Heron's method that's
more efficient than the basic algorithm for large inputs. As in
the previous section, we assume that~$n$ is a positive integer, and we aim to
compute the integer square root of~$n$.

There are two key ideas. First, we vary the working precision as we go: our
algorithm produces at each iteration an integer approximation to the square
root of $\floor{n/4^f}$ for some integer $f$, with $f$ decreasing to $0$ as the
iterations progress. Second, we don't insist on obtaining the exact integer
square root of $\floor{n/4^f}$ at each iteration (which would require a
per-iteration check-and-correct step), but instead allow the error to
propagate, and we prove that with careful control of the rate at which the
precision increases, the error remains bounded throughout the algorithm. We
then use a single check-and-correct step at the end of the algorithm.

To describe the algorithm, it's convenient to introduce a new notion: that of a
``near square root".

\begin{definition}
  Suppose $n$ is a positive integer. Call a positive integer $a$ a
  \emph{near square root} of $n$ if $(a - 1)^2 < n < (a + 1)^2$.
\end{definition}

In other words, $a$ is a near square root of $n$ if $a$ is either $\floor{\sqrt
n}$ or $\ceil{\sqrt n}$. In particular, if $n = a^2$ is a perfect square then
$a$ is the \emph{only} near square root of $n$.

Given a near square root $a$ of $n$, the integer square root of $n$ is clearly
either $a$ or $a-1$, depending on whether $a^2 \le n$ or $a^2 > n$
(respectively). So an algorithm for computing near square roots provides us
with a way to compute integer square roots, and in the remainder of this
section we focus on finding near square roots.

Our algorithm is based on the idea of ``lifting" a near square root of a
quotient of $n$ to a near square root of $n$. Suppose that $j$ is a positive
integer and $b$ is a near square root of $\floor{n / j^2}$. Then $b$ is an
approximation to $\sqrt n / j$, and so $jb$ is an approximation to $\sqrt n$. A
single iteration of the integer form of Heron's method applied to $jb$ then
gives an improved approximation
$$\floor*{\frac{jb + \floor{n / jb}}2} \sim \sqrt n.$$
If $j$ is not too large relative to~$n$, this
improved approximation will again be a near square root of~$n$.

Here's a theorem that makes that ``not too large" bound precise. For
simplicity, we restrict~$j$ to be an even integer: write $j = 2k$, then in the
above discussion, $b$ is a near square root of~$\floor{n / 4k^2}$, $2kb$ is an
approximation to $\sqrt n$, and our improved approximation is $kb + \floor{n /
4kb}$.

\begin{theorem}\label{main_theorem}
  Suppose that $n$ and $k$ are positive integers satisfying $4k^4 \le n$. Let
  $b$ be a near square root of $\floor{n / 4k^2}$. Then the positive integer
  $a$ defined by
  $$a = kb + \floor*{\frac{n}{4kb}}$$
  is a near square root of $n$.
\end{theorem}

\begin{proof}
  By definition of near square root, we have
  \begin{equation}\label{b_nsqrt}
    (b - 1)^2 < \floor*{\frac{n}{4k^2}} < (b + 1)^2.
  \end{equation}
  Since $(b + 1)^2$ is an integer, we can remove the floor brackets to obtain
  \begin{equation}\label{b_nsqrt_weak}
    (b - 1)^2 < \frac{n}{4k^2} < (b + 1)^2.
  \end{equation}
  Multiplying by $4k^2$ throughout \eqref{b_nsqrt_weak} and then taking square
  roots gives
  \begin{equation}
    2k(b - 1) < \sqrt n < 2k(b + 1)
  \end{equation}
  which can be rearranged to the equivalent statement
  \begin{equation}
    \abs{2kb - \sqrt n} < 2k.
  \end{equation}
  Squaring and then dividing through by $4kb$ gives
  \begin{equation}
    0 \le kb + \frac{n}{4kb} - \sqrt n < k/b,
  \end{equation}
  which implies that
  \begin{equation}
    -1 < kb + \floor*{\frac{n}{4kb}} - \sqrt n < k/b.
  \end{equation}
  Substituting the definition of $a$ gives
  \begin{equation}\label{a_close_to_sqrt_n}
    -1 < a - \sqrt n < k/b.
  \end{equation}
  To complete the proof, we need to know that $k / b \le 1$. From our (so far
  unused) assumption that $4k^4 \le n$ we have $k^2 \le n / 4k^2$, while from
  the right-hand side of \eqref{b_nsqrt_weak} we have $n / 4k^2 < (b + 1)^2$.
  Combining these and taking square roots, $k < b + 1$, hence $k \le b $ and $k
  / b \le 1$. So now combining this with \eqref{a_close_to_sqrt_n} gives
  \begin{equation}
    -1 < a - \sqrt n < 1
  \end{equation}
  from which $(a - 1)^2 < n < (a + 1)^2$, so $a$ is a near square of~$n$, as
  required.
\end{proof}

\begin{example}
  Let $n = 46696$ and $k=10$. Then $\floor{n / k^2} = 116$, so $10$ and
  $11$ are both near square roots of $\floor{n / k^2}$.

  Taking $b = 10$, we get $a = 100 + \floor{46696 / 400} = 216$. Since
  $\sqrt{46696} = 216.092572755\dots$, $216$ is indeed a near square root of
  $n$. If we take $b = 11$, we also get $a = 110 + \floor{46696 / 440} = 216$.
\end{example}

Now we turn to implementation. Like many arbitrary-precision integer
implementations, Python's integer implementation is binary-based, so
multiplications and floor divisions by powers of two can be performed
efficiently by bit-shifting. So when applying Theorem~\ref{main_theorem}, we'll
take for our $k$ the largest power of two satisfying $4k^4 \le n$. Writing
$k=2^e$, we have:
\begin{align*}
  4k^4 \le n
  &\iff 2^{2 + 4e} \le n \\
  &\iff 2 + 4e < \length n \\
  &\iff 3 + 4e \le \length n \\
  &\iff e \le \floor*{\frac{\length n - 3}{4}}
\end{align*}
So we take $k = 2^e$, where $e={\floor{(\length n - 3)/4}}$. This gives the
recursive near square root implementation shown in
Listing~\ref{isqrt_recursive}, where the multiplication by $k$ and the
divisions by $4k$ and $4k^2$ are replaced by the corresponding bit-shift
operations.

\lstinputlisting[
  float,
  frame=single,
  label=isqrt_recursive,
  caption={Adaptive precision Heron's method, recursive},
  lastline=11]{isqrt_recursive.py}

Each step of the algorithm involves three big-integer shifts, one
big-integer addition, one bit-length computation, and one big-integer division,
along with a handful of operations that only involve small integers.
We can improve this slightly: one of the two right-shifts can be eliminated, by
keeping track of the amount by which~$n$ should be shifted in the recursive
call instead of actually shifting. With a little more bookkeeping, we can
also replace the per-iteration bit-length computation with a single
initial bit-length computation. These changes represent minor efficiency
improvements at the expense of some small loss of clarity; we leave the
interested reader to pursue them further.

The \lstinline{math.isqrt} implementation in CPython 3.8 doesn't use the
recursive version shown in Listing~\ref{isqrt_recursive}. Instead, we unwind
the recursion to obtain an iterative version of the algorithm. This version is
presented in Listing~\ref{isqrt_iterative}.

\lstinputlisting[
  float,
  frame=single,
  label=isqrt_iterative,
  caption={Adaptive precision Heron's method, iterative},
  lastline=13]{isqrt_iterative.py}

It may not be immediately obvious that Listing~\ref{isqrt_iterative} is
equivalent to Listing~\ref{isqrt_recursive}. Rather than describing the
equivalence and establishing the correctness of the iterative version via that
equivalence, it's simpler to give a direct proof of correctness for the
iterative version.

We establish two loop invariants on the variables $s$, $d$ and $a$. These
invariants hold just before entering the \lstinline$while$ loop, at the end
of every iteration of that loop, and hence also the beginning of any subsequent
iteration.

The first loop invariant is $d = \floor{c/2^s}$; this should be clear from
examining the code. The second loop invariant states that at every step, $a$ is
a near square root of $\floor{n / 4^{c-d}}$. In particular, on exit from the
\lstinline$while$ loop, $s = 0$, $d = \floor{c/2^0} = c$ and so $a$ is a near
square root of $\floor{n / 4^{c-c}} = n$.

To establish the second invariant, note first that the invariant holds on
entry to the \lstinline$while$ loop: we have $c = \floor{\log_4 n}$, so
$4^c \le n < 4^{c+1}$ and $1 \le \floor{n / 4^c} < 4$, so $a = 1$ is a near
square root of $\floor{n / 4^c}$. We then need to establish that if the
invariant holds at the beginning of any while loop iteration, it also applies
at the end of that iteration. We prove this via the following lemma, in which
$b$ and $e$ represent the values of $a$ and $d$ at the start of the iteration.

\begin{lemma}
  Suppose that $0 \le s < \length(c)$, that $e = \floor{c / 2^{s + 1}}$, that
  $d = \floor{c / 2^s}$, and that $b$ is a near square root of $\floor{n /
  4^{c-e}}$. Let
  $$a = 2^{d-e-1}b + \floor*{\frac{n}{2^{2c-d-e+1}b}}.$$ Then $a$ is a near
  square root of $\floor{n / 4^{c-d}}$.
\end{lemma}

\begin{proof}
  Let $m = \floor{n / 4^{c-d}}$. Then $b$ is a near square root of
  $\floor{m/4^{d-e}}$ and we can rewrite $a$ as
  $$a = 2^{d-e-1}b + \floor*{\frac{m}{2^{d-e+1}b}}.$$ Now we can apply the main
  theorem: if we can show that $2^{d-e-1}$ is an integer and that
  $4(2^{d-e-1})^4 \le m$, it follows from Theorem~\ref{main_theorem} that $a$
  is a near square root of $m$. But from the definitions of $d$ and $e$, $1 \le
  d$ and $e = \floor{d/2}$, so it follows that $0 \le d - e - 1 \le e$, and
  \begin{align*}
    4(2^{d-e-1})^4 &= 4(4^{d-e-1})^2 \\
                  &= 4\cdot 4^{d-e-1}\cdot 4^{d-e-1} \\
                  &\le 4\cdot 4^{d-e-1}\cdot 4^e \\
                  &= 4^d
  \end{align*}
  Furthermore, since $c = \floor{\log_4 n}$, $4^c \le n$, so $4^d \le n /
  4^{c-d}$, hence $4^d \le m$. Combining this with the inequality above,
  $4(2^{d-e-1})^4 \le m$, as required.
\end{proof}

The quantities $c$, $d$, $e$, and $s$ are all small (machine-size) integers;
only $a$ and $n$ are big integers. So the algorithm consists of two big-integer
shifts, one big-integer addition and one big-integer division per iteration,
along with a handful of operations with small integers.

The total number of iterations $m$ of the while loop turns out to be remarkably
simple: it's exactly $\floor*{\log_2 \floor{\log_2 n}}$, assuming $n\ge 2$ (and
zero iterations are required for $n = 1$). So for example, input values $n$
satisfying $2^{32} \le n < 2^{64}$ require exactly five iterations, while
values in the range $2^{64} \le n < 2^{128}$ require exactly six.

\section{Fixed-precision algorithms}

The iterative algorithm shown in Listing~\ref{isqrt_iterative} specialises
easily to particular fixed-precision cases, giving an almost branch-free
algorithm. For example, if $2^{30} \le n < 2^{32}$, then $c = 15$, $s=4$, and
we can compute in advance the set of $d$ and $e$ values for each of the four
iterations. For smaller positive~$n$, we can find an $f$ such that $2^{30} \le
4^f n < 2^{32}$, apply the same algorithm to $4^f n$, then shift the resulting
near square root right by~$f$ bits. This gives the algorithm shown in
Listing~\ref{isqrt_32bit}. The corresponding 64-bit version is shown in
Listing~\ref{isqrt_64bit}. While the original iterative algorithm was developed for
positive~$n$, these fixed-precision variants also turn out to give the correct
answer in the case~$n = 0$.

\lstinputlisting[
  float,
  frame=single,
  label=isqrt_32bit,
  caption={Fixed-precision variant, valid for $0 \le n < 2^{32}$},
  lastline=10]{isqrt_32bit.py}

\lstinputlisting[
  float,
  frame=single,
  label=isqrt_64bit,
  caption={Fixed-precision variant, valid for $0 \le n < 2^{64}$},
  lastline=10]{isqrt_64bit.py}


\section{Using floating-point}

If we have access to a binary floating-point type that provides IEEE 754 format
and semantics, we can use floating-point arithmetic to compute integer square
roots directly for small~$n$, and to accelerate the computation of integer
square roots for larger~$n$. Note that Python does not currently require IEEE
754 format or semantics, but that in practice use of IEEE 754 floating-point is
almost ubiquitous on platforms that support Python, and Python's
\lstinline{float} type is highly likely to map to the IEEE 754
``double precision'' binary64 interchange format, defined in section~3.6 of
IEEE 754-2019.

In this section, we assume that Python's \lstinline{float} type \emph{does} use
the binary64 format, and that basic arithmetic operations and the standard
library \lstinline{math.sqrt} function are all correctly rounded, using the
default IEEE 754 ``roundTiesToEven" rounding mode.

First some definitions: given a real number~$x$, write $\float(x)$ for the
nearest (under ``roundTiesToEven") binary64 floating-point number to~$x$. If
$x$ has magnitude $2^{1024} - 2^{970}$ or greater, $\float(x)$ is the
appropriately-signed infinity. If $x$ is already a finite floating-point
number, we implicitly regard it as a real number when necessary. Given a finite
nonnegative binary64 floating-point number $x$, write $\fsqrt(x)$ for the
correctly-rounded square root of~$x$, and $\fsqr(x)$ for the correctly-rounded
square of $x$. In other words, $\fsqrt(x) = \float(\sqrt x)$ and $\fsqr(x) =
\float(x^2)$.

We first prove the statement given in the introduction, that if $n$ is
not too large, $\floor{\fsqrt(n)}$ gives the integer square root of~$n$.

\begin{lemma}
  Suppose that $n$ is a nonnegative integer satisfying $n < 2^{52}$. Then
  $\floor{\fsqrt(n)}$ is the integer square root of~$n$.
\end{lemma}

\begin{proof}
  The lemma is clearly true for~$n=0$, so for the remainder of the proof
  we assume that~$n$ is positive.

  Write $a$ for the integer square root of $n$. Then $a^2 \le n < (a + 1)^2$,
  so $a \le \sqrt n < a + 1$, and since both $a$ and $a + 1$ are small
  enough to be exactly representable as floats, and the $\float$ operation
  is monotonic, it follows that
  $$a = \float(a) \le \float(\sqrt n) \le \float(a + 1) = a + 1.$$
  So
  $$a \le \fsqrt(n) \le a + 1,$$
  and to prove the lemma, it suffices to eliminate
  the possibility that~$\float(\sqrt n) = a + 1$.

  Since~$n$ is positive, so is~$a$, and we can find an integer $k$ such that
  $2^k \le a < 2^{k + 1}$. Any two consecutive floating-point numbers in the
  interval $[2^k, 2^{k+1}]$ are separated by exactly $2^{k-52}$, so the largest
  floating-point number that's strictly smaller than $a + 1$ is $a + 1 - 2^{k -
  52}.$ To avoid the possibility that $\sqrt n$ rounds up to $a + 1$, it's
  enough to show that $\sqrt n$ is closer to $a + 1 - 2^{k-52}$ than to~$a +
  1$; that is, that
  $$\sqrt n < a + 1 - 2^{k-53}$$
  or equivalently that
  $$a + 1 - \sqrt n > 2^{k - 53}.$$
  But we have
  $$a + 1 - \sqrt n = \frac{(a + 1)^2 - n}{a + 1 + \sqrt n}.$$
  The numerator on the right-hand side above is at least~$1$, and since $\sqrt
  n < a + 1$, the denominator is less than $2(a+1)$, which is in turn less than
  or equal to $2^{k+2}$. So
  $$a + 1 - \sqrt n > 2^{-k-2}.$$ So we only need to show that $2^{-k-2} \ge
  2^{k-53}$. But that's equivalent to $2k \le 51$, and from our assumptions,
  $a^2 \le n < 2^{52}$, so $2^k \le a < 2^{26}$, hence $k < 26$.
\end{proof}


\begin{lemma}\label{float_scale}
  Suppose that $x$ and $y$ are positive real numbers such that $x = 2^e y$ for
  some integer $e$, and that both $x$ and $y$ lie in the half-open interval
  $[2^{-1022} - 2^{-1076}, 2^{1024} - 2^{970})$. Then $\float(x) = 2^e
  \float(y)$.
\end{lemma}

\begin{proof}
  The bounds on $x$ and~$y$ ensure that $\float(x) = 2^f \rint(x / 2^f)$, where
  $f = \floor{\log_2 x} - 52$, and similarly for~$y$. Here $\rint(x)$ is the
  operation that rounds a real number to the nearest integer, rounding halfway
  cases to the even integer. The result follows directly from this.
\end{proof}

The following fact is well known, but we state and prove it here for
convenience.

\begin{theorem}\label{sqrt_of_sqr}
  Suppose $x$ is an IEEE 754 binary64 floating-point number satisfying
  $2^{-511} \le x < 2^{512}$. Then $\fsqrt(\fsqr(x)) = x$.
\end{theorem}

\begin{proof}
  We first use the previous lemma to reduce to the case where $1/2 < x \le 1$.
  Choose an integer $e$ and floating-point number $y$ such that $1/2 < y \le 1$
  and
  $$x = 2^e y.$$
  Then
  $$x^2 = 2^{2e}y^2$$
  so applying Lemma~\ref{float_scale},
  $$\float(x^2) = 2^{2e}\float(y^2).$$
  By definition of $\fsqr$, this can be rewritten as
  $$\fsqr(x) = 2^{2e}\fsqr(y).$$
  Now taking square roots of both sides,
  $$\sqrt{\fsqr(x)} = 2^e \sqrt{\fsqr(y)}.$$
  Applying Lemma~\ref{float_scale} again,
  $$\float(\sqrt{\fsqr(x)}) = 2^e \float(\sqrt{\fsqr(y)}),$$
  or in other words
  $$\fsqrt(\fsqr(x)) = 2^e \fsqrt(\fsqr(y)).$$
  So to prove that $\fsqrt(\fsqr(x)) = x$, it's enough to prove that
  $\fsqrt(\fsqr(y)) = y$, since we then have
  $$\fsqrt(\fsqr(x)) = 2^e \fsqrt(\fsqr(y)) = 2^e y = x.$$
  Or in other words, it's enough to prove the original statement in the special
  case that $1/2 < x \le 1$. So from this point on, we assume that $1/2 < x \le
  1$.

  Let $z = \fsqr(x)$ be the closest float to $x^2$. Then $1/4 \le z \le 1$ and
  since floats are spaced at most $2^{-53}$ apart within the interval $[1/4,
  1]$,
  $$\abs{z - x^2} \le 2^{-54}.$$ Now we have
  $$\sqrt z - x = \frac{(\sqrt z - x)(\sqrt z + x)}{\sqrt z + x} = \frac{z -
  x^2}{\sqrt z + x}$$ and since $1/2 < x$ and $1/2 \le \sqrt z $, $\sqrt z + x
  > 1$. It follows that
  $$\abs{\sqrt z - x} < 2^{-54}$$ and so since floats in the interval $[1/2,
  1]$ are spaced exactly $2^{-53}$ apart, it follows that $x$ is the unique
  closest float to $\sqrt z$, so $x = \fsqrt(z)$ as required.
\end{proof}

The next corollary follows immediately from Theorem~\ref{sqrt_of_sqr}, together
with the fact that any nonnegative integer not exceeding $2^{53}$ can be
exactly represented in binary64 floating-point.

\begin{corollary}
  Suppose $n$ is a nonnegative integer satisfying $n \le 2^{53}$. Then
  $\fsqrt(\fsqr(n)) = n$.
\end{corollary}

Now we're in a position to prove that for sufficiently small~$n$, the
floating-point square root can be used to compute a near square root of~$n$.

\begin{corollary}
  Suppose $n$ is a positive integer satisfying $n \le 2^{106}$. Then
  $\floor{\fsqrt(\rnd(n))}$ is a near square root of $n$.
\end{corollary}

\begin{proof}
  First suppose that $n$ is a perfect square: $n = a^2$. Then $\rnd(n) =
  \fsqr(a)$, so from the previous corollary, $\fsqrt(\rnd(n)) =
  \fsqrt(\fsqr(a)) = a$.

  Now suppose that $n$ is not a perfect square, so that $a^2 < n < (a+1)^2$
  for some nonnegative integer $a < 2^{53}$. Then $\rnd$ is monotonic, so
    $$\rnd(a^2) \le \rnd(n) \le \rnd((a+1)^2)$$
  or equivalently,
    $$\fsqr(a) \le \rnd(n) \le \fsqr(a+1).$$
  Similarly, $\fsqrt$ is monotonic, so applying $\fsqrt$ throughout
  and using the previous corollary,
    $$a \le \fsqrt(\rnd(n)) \le a + 1.$$
  Hence $\floor{\fsqrt(\rnd n)}$ is either $a$ or $a+1$, so $a$ is a near
  square root of $n$.
\end{proof}

In fact, $\floor{\fsqrt(\rnd n)}$ gives a near square root of~$n$ for all $n
\le 2^{106} + 2^{54}$. The smallest~$n$ for which it fails to give a near
square root is $n = 2^{106} + 2^{54} + 1 = (2^{53} + 1)^2$. For this~$n$, $\rnd
n = 2^{106} + 2^{54}$ and $\fsqrt(\rnd n) = 2^{53}$.

Listing~\ref{isqrt_accelerated} shows a version of the iterative
algorithm in Listing~\ref{isqrt_iterative}, modified to use floating-point
arithmetic to compute the initial value of~$a$.  If~$n < 2^{106}$,
$a$ is already a near square root of~$n$ and the \lstinline$while$ loop is
executed zero times. More generally, the number of iterations required
is~$\floor{\log_2((\log_2n)/53)}$ for~$n \ge 2^{53}$, and~$0$ otherwise.


\lstinputlisting[
  float,
  frame=single,
  label=isqrt_accelerated,
  caption={Integer square root using floating-point quick start}]
  {isqrt_accelerated.py}


% \section{To do}

% - Note that if a value is already known to be a square, no check-and-correct
% step is needed.
% - Similarly for square detection: take a near square root, square it, and check.
% - And square root ceiling is easy, too.
% - Running time analysis, and real-world timings to back it up. Asymptotically,
% isqrt of a $2n$-bit integer should be faster than division of a $2n$-bit integer
% by an $n$-bit dividend.
% - Fixed-precision versions.
% - Floating-point methods and floating-point accelerations.
% - Details on correctness of floating-point algorithm.
% - Reference(s) for "Many number-theoretic and cryptographic algorithms ..." Cohen?
% - Reference for Java BigInteger.sqrt
% - Reference for Heron's method / Babylonian method / Newton's method.
% - Reference for assertion that the number of correct decimal places
% doubles with each iteration.

\end{document}
